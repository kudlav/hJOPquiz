\documentclass[12pt,a4paper]{article}
\usepackage{xltxtra}
\usepackage{polyglossia}
\setdefaultlanguage{czech}
\defaultfontfeatures{Mapping=tex-text}
\setmainfont{Carlito} % Cambria on linux
% \setmonofont{Consolas} % Uncommon font for linux
\def\uv#1{„#1“}

\usepackage[ unicode
           , pdfauthor={Jan Horáček}
           , pdftitle={hJOPquiz},
           , pdfsubject={Kvíz testující schopnost řídit hJOP},
           , plainpages=false
           , pdfpagelabels
           , draft=false
           , colorlinks=false
           , unicode=true
           ]{hyperref}
\usepackage{graphicx}

\usepackage{microtype}
\usepackage{enumitem}
\usepackage{titling}
\usepackage{xcolor}
\textwidth 16cm \textheight 24.6cm
\topmargin -1.3cm
\oddsidemargin 0cm

\def\solution#1{\ifsolution \par\vskip-\parskip{\color{gray}#1}\fi}

\def\symbol#1#2#3{ % filename, scale, description
    \ifsolution
       \parbox[t]{.305\linewidth}{\parskip 0pt\centering\includegraphics[width=#2\textwidth]{#1}\par
       \vskip -5pt\vrule height 0pt depth 3\baselineskip width 0pt\vtop{\color{gray}
                #3\vfill}}\quad
    \else
       \includegraphics[width=#2\textwidth]{#1}%
    \fi
}


\begin{document}
\thispagestyle{empty}

\setlength{\parindent}{0cm}
\setlength{\parskip}{.4\baselineskip plus2pt minus1pt}
\setlength{\droptitle}{-5em}

\title{\bfseries
{\Large Ovládání kolejiště pomocí hJOP\\}
{\LARGE Podklady pro zkoušku oprávnění řízení\\
bez oprávnění regulátoru\\}
\ifsolution
{\Large \color{gray}včetně řešení\\}
\fi
{\small v2.0}}
\author{Jan Horáček (jan.horacek@kmz-brno.cz)}
\date{\today}
\maketitle

Tento dokument obsahuje podklady pro učení a testování způsobilosti řízení
hJOP. Jeho cílem je nastavit jednotnou míru znalostí obsluhy kolejiště, a~tím
předejít zbytečným chybám, zdržování provozu, újmám na majetku a~přispět
k~obecně lepší součinnosti mezi obsluhou kolejiště.

Následující text obsahuje několik úrovní zaškolení – od základní až po tu
nejpokročilejší. Každá úroveň obsahuje teoretické otázky a~praktickou část, ve
které probíhá řešení zadaných úkolů přímo na kolejišti. Každý, kdo chce získat
příslušné oprávnění, musí odpovědět na všechny otázky a~být schopen vyřešit
všechny praktické problémy. Otázky v~jednotlivých sekcích jsou seřazeny od
nejdůležitějších k~méně důležitým.

\section{Doporučená metodika zaškolování}

\begin{enumerate}[leftmargin=*]
\item Nováčka zařadit k~někomu (patronovi) na úrovni, kterou požaduje, ten ho
naučí základy obsluhy.
\item Nováčkovi dát k~dispozici tento dokument, on si přečte dotazy, případné
nejasnosti vyřeší se zaškoleným patronem.
\item Oprávnění získává nováček výhradně u~školitele – u~kohokoliv s~oprávněním
A.
Školitel vybere otázky z~tohoto dokumentu a~nechá nováčka odpovědět. Školitel
provede s~no\-váč\-kem praktickou část zkoušky.
\item Školitel přidává login nováčka do hJOPserveru, nováček je oprávněn řídit.
\end{enumerate}

\newpage

\section{S1 – dispečer základní}

Opravňuje k~řízení menších stanic.

Pro tuto úroveň zaškolení je doporučeno mít zaškolení na úrovni S0. Výjimkou
je situace, kdy dispečeři používají k~řízení jízdy jiné prostředky než mobilní
aplikaci hJOPdriver.

\subsection{Praktická část}

\subsubsection*{Přihlášení, odhlášení}
\begin{enumerate}[leftmargin=*]
\item Spusťte panel. Připojte se k~serveru, přihlaste se, odpojte se od
serveru. Přihlaste se v~režimu čtenáře, demonstrujte přihlášení více panelů.
\item Proveďte přihlášení k~jednomu panelu a ke všem spuštěným panelům.
\item Zobrazte a skryjte popisky bloků.
\end{enumerate}

\subsubsection*{Základní obsluha vlaků}
\begin{enumerate}[leftmargin=*]
\item Obslužte několik vlaků.
\item Proveďte křižování osobních vlaků.
\end{enumerate}

\subsubsection*{Dopravní kancelář}
\begin{enumerate}[leftmargin=*]
\item Demonstrujte komunikaci se sousedními dispečery (telefon, poslání zprávy).
\item Demonstrujte přepnutí řízení na místní a dálkový provoz.
\end{enumerate}

\subsubsection*{Soupravy}
\begin{enumerate}[leftmargin=*]
\item Proveďte vytvoření soupravy.
\item Proveďte smazání soupravy.
\item Zobrazte seznam souprav v~obvodech všech vámi řízených stanic, vysvětlete
význam tohoto seznamu.
\item Upravte výchozí a cílovou stanici soupravy v~koncové stanici.
\end{enumerate}

\subsubsection*{Bloky}
\begin{enumerate}[leftmargin=*]
\item Proveďte ruční přestavení výhybky.
\item Proveďte nouzové ruční přestavení výhybky.
\item Proveďte nouzové uvolnění závěrů úseků. Vysvětlete, kdy se takové uvolnění
provádí.
\item Demonstrujte obsluhu rozpojovačů.
\item Zaveďte štítek na úseku.
\end{enumerate}

\subsubsection*{Krizové scénáře}
\begin{enumerate}[leftmargin=*]
\item Proveďte nouzové zastavení provozu na celém kolejišti.
\end{enumerate}

\subsubsection*{Poznání}
\begin{enumerate}[leftmargin=*]
\item Demonstrujte poznání trati, ukažte polohy jednotlivých návěstidel, typy
traťových zabezpečovacích zařízení.
\end{enumerate}


\subsection{Teoretická část}

\subsubsection*{Přihlašování, odhlašování}
\begin{enumerate}[leftmargin=*]

\item Vysvětlete, jak funguje mechanismus přihlašování k~více panelům zároveň.
Kdy tento mechanismus použít a~kdy naopak ne?
\solution{Pro přihlášení k~více panelům zároveň je nutné nejdříve spustit
všechny požadované panely, pak se v~jednom z~nich připojit k~serveru a~ponechat
zatrhnutou možnost \textit{Autorizovat další panely}. Tím se automaticky
přihlásí všechny další spuštěné panely. \\ Volbu \textit{Autorizovat další
spuštěné panely} je vhodné ručně odtrhnout v~případě, kdy se chcete přihlásit
pouze k~jedné stanici, například jako čtenář. Pokud byste tuto volbu ponechali
zatrhnutou, všechny ostatní spuštěné panely, na které můžete být přihlášeni vy,
se přehlásí na čtenáře, a~vy tak ztratíte možnost ovládat kolejiště.}

\item Jak poznáte, že je panel připojen k~serveru?
\solution{Vlevo dole na panelu je napsáno \textit{Připojeno k~serveru}}.

\item Vysvětlete význam přepnutí stanice na místní a dálkový provoz.
\solution{Místní provoz (šedá DK) = ovládáte stanici, ovládáte pouze vy a~nikdo
jiný. Dálkový provoz (bílá DK) = pozorujete stanici, neovládáte, stanici může
ovládat někdo jiný.}

\item Popište postup přebírání stanice jiným dispečerem, například při změně
směn.
\solution{Odcházející dispečer se odhlásí, nový dispečer se přihlásí na svůj
účet.}

\item Chcete se přihlásit ke stanici pouze na dívání, jak to provedete?
\solution{Při přihlašování zvolíte \textit{Přihlásit se jako host} a ideálně
odškrtnete \textit{Autorizovat další panely}.}

\item Jak potlačíte zvukovou notifikaci zkratu na kolejišti? Je toto
potlačení trvalé?
\solution{Na horní liště je ikona s~reproduktorem a křížkem. Potlačení trvá
1~minutu.}

\item Můžete z~libovolného pracoviště ovládat libovolnou stanici?
\solution{Pokud je ikona na ploše, ano.}

\item Jakými všemi metodami mohu komunikovat s~dispečery sousedních stanic?
\solution{Napsání zprávy, zavolání telefonem, fyzický rozhovor.}

\item Ke kolejišti přijde váš kamarád, který si chce zajezdit, ale nemá
zaškolení (nemá login), co mu můžete povolit a co naopak nesmíte? Jak se k~němu
máte chovat?
\solution{Kamarád nesmí samostatně řídit stanici. Můžete ho pustit k~počítači
a~nechat ho provádět úkony dispečera, ale musí být pod vaším trvalým dohledem.}

\end{enumerate}

\subsubsection*{Soupravy}
\begin{enumerate}[leftmargin=*]
\item Jaký je rozdíl mezi vlakem, lokomotivou a~soupravou? Může se lokomotiva
sama pohybovat na širé trati?
\solution{Vlak = souprava. Souprava je složena z~jedné nebo více lokomotiv.
Lokomotiva se na trati typicky pohybuje jako lokomotivní vlak.}

\item Jak poznáte typ blížícího se vlaku? Jaká rozhodnutí na základě typu
typicky děláte?
\solution{Z~předčíslí soupravy. Podle typu vlaku například volíte kolej, na
kterou vlak přijmete: osobní vlaky vždy k~perónu.}

\item Vyjmenujte, jaká předčíslí souprav mají jaký význam.

\item Jaké je typické složení čísla vlaku? Jak poznat z~čísla vlaku DCC adresu
lokomotivy?
\solution{Šestimístné číslo vlaku: dvojmístné předčíslí (typ soupravy)
a~čtyřmístná adresa dekodéru hnacího vozidla.}

\item Vysvětlete rozdíl mezi možným směrem vlaku a~směrem stanoviště A.
\solution{Směr vlaku udává možný směr, kterým se vlak může pohybovat. Může být
jeden nebo i~oba směry. Podle možného směru vlaku se vykresluje šipka nad
číslem vlaku v~reliéfu. Možný směr vlaku neovlivňuje fyzický směr jízdy
lokomotivy. Možný směr vlaku je vlastnost vlaku. \\ Orientace stanoviště A~je
vlastnost lokomotivy. Udává, jakým směrem je lokomotiva fyzicky na kolejišti.
Je nutné jej správně zadat, aby lokomotiva jela správným směrem.}

\item Jak poznat, že mají hnací vozidla vlaku správně zadána stanoviště A?
\solution{Všechny lokomotivy vlaku svítí správným směrem.}

\item Kdy je a~kdy není třeba upravovat výchozí a~cílovou stanici soupravy?
\solution{Výchozí a~cílová stanice se automaticky mění ve smyčkách. Pokud vlak
končí v~jiné stanici, je třeba mu upravit výchozí a~cílovou stanici ručně.
Vlak, který dorazit do cílové stanice, má své číslo šedě podbarvené.}

\item Kolik lokomotiv může mít souprava?
\solution{Až 4.}

\item Je nutné měnit orientaci stanoviště A~při otáčení vlaku v~koncové
stanici?
\solution{Ne. Naopak, dělat se to nesmí!}

\item Je nutné zadávat vlaku jeho délku a~typ? Proč?
\solution{Ano. Pro správné zastavování v~zastávkách a~u~perónů ve stanicích.}

\item Na které místo ve vlaku je možné přivěsit čisticí vůz?
\solution{Výhradně hned za lokomotivu.}

\end{enumerate}

\subsubsection*{Bloky}
\begin{enumerate}[leftmargin=*]
\item Jak poznáte zkrat na kolejišti? Co s~ním dělat?
\solution{Fialové podbarvení úseků. Zkrat je typicky způsoben najetím do špatně
přestavené výhybky: ručně zastavte lokomotivu a~na panelu nouzově přestavte
výhybku.}

\item Vysvětlete význam následujících symbolů na reliéfu. \\
\symbol{symboly/kol1.png}{0.1}{Kolej s~prostředky pro kontrolu volnosti volná.}
\symbol{symboly/kol2.png}{0.1}{Kolej obsazená.}
\symbol{symboly/kol3.png}{0.1}{Kolej volná, závěr vlakové cesty.}
\symbol{symboly/kol4.png}{0.1}{Kolej volná, závěr posunové cesty.}
\symbol{symboly/kol22.png}{0.1}{Kolej volná, zkrat.}
\symbol{symboly/kol8.png}{0.1}{Kolej volná, výpadek DCC.}

\symbol{symboly/hlnav16.png}{0.1}{Hlavní návěstidlo nekomunikuje.}
\symbol{symboly/hlnav1.png}{0.1}{Základní stav.}
\symbol{symboly/hlnav2.png}{0.1}{Povolující návěst pro vlak (mimo PN).}
\symbol{symboly/hlnav3.png}{0.1}{Povolující návěst pro posun. Přerušovaně:
přivolávací návěst.}
\symbol{symboly/hlnav5.png}{0.1}{Návěstidlo mění návěst / návěstidlo zhaslé.}
\symbol{symboly/hlnav6.png}{0.1}{Zamknuto do návěsti stůj.}
\symbol{symboly/hlnav8.png}{0.1}{Základní stav + volba vlakové cesty.}

\symbol{symboly/senav6.png}{0.1}{Seřaďovací návěstidlo nekomunikuje.}
\symbol{symboly/senav2.png}{0.1}{Povolující návěst.}
\symbol{symboly/senav3.png}{0.1}{Návěstidlo mění návěst / návěstidlo zhaslé.}
\symbol{symboly/senav4.png}{0.1}{Zamknuto do návěsti posun zakázán.}
\symbol{symboly/senav8.png}{0.1}{Základní stav + volba posunové cesty.}

\symbol{symboly/vyh4.png}{0.1}{Ztráta komunikace.}
\symbol{symboly/vyh1.png}{0.1}{Přímý směr + úsek volný.}
\symbol{symboly/vyh2.png}{0.1}{Odbočný směr + úsek volný.}
\symbol{symboly/vyh3.png}{0.1}{Úsek volný + ztráta dohledu.}
\symbol{symboly/vyh5.png}{0.1}{Úsek obsazen.}
\symbol{symboly/vyh6.png}{0.1}{Úsek obsazen + ztráta dohledu.}
\symbol{symboly/vyh28.png}{0.1}{Výpadek napájení zesilovače (napěťová výluka) +
závěr vlakové cesty.}
\symbol{symboly/vyh37.png}{0.1}{Výhybka nevybavená zařízením pro kontrolu polohy.}
\symbol{symboly/vyh38.png}{0.1}{S~kontrolou volnosti kolejového úseku + úsek obsazen.}
\symbol{symboly/vyh39.png}{0.1}{Bez kontroly volnosti kolejového úseku.}

\symbol{symboly/vyk19.png}{0.1}{Ztráta komunikace.}
\symbol{symboly/vyk1.png}{0.1}{Na koleji + úsek volný.}
\symbol{symboly/vyk2.png}{0.1}{Sklopená + úsek volný.}
\symbol{symboly/vyk3.png}{0.1}{Na koleji + úsek obsazen.}
\symbol{symboly/vyk4.png}{0.1}{Sklopená + úsek obsazen.}
\symbol{symboly/vyk5.png}{0.1}{Ztráta dohledu + úsek volný.}
\symbol{symboly/vyk6.png}{0.1}{Ztráta dohledu + úsek obsazen.}

\symbol{symboly/ez6.png}{0.05}{Ztráta komunikace.}
\symbol{symboly/ez1.png}{0.05}{Klíč zapevněn.}
\symbol{symboly/ez3.png}{0.05}{Klíč vyjmut.}
\symbol{symboly/ez5.png}{0.05}{Ztráta kontroly.}

\symbol{symboly/prej1.png}{0.1}{Otevřen.}
\symbol{symboly/prej2.png}{0.1}{Uzavřen povelem dispečera.}
\symbol{symboly/prej3.png}{0.1}{Nouzově otevřen.}

\item Kdy je a~kdy není nutné žádat o~traťový souhlas?
\solution{O~traťový souhlas není nutné žádat, pokud ho mám udělen, nebo pokud
jsem přihlášený do obou stanic tratě.}

\item Co je to závěr a k~čemu slouží?
\solution{Závěr je zapevnění bloku proti změně. Typicky například zamknutí
výhybky. Závěr se uděluje na bloky v~jízdní cestě. Na výhybky, výkolejky a
úvazky lze ručně udělit nouzový závěr volbou \texttt{ZAV>} v~menu bloku.}

\item Sousední stanice vás žádá o~traťový souhlas, vy ho však chcete přijmout až
za minutu, jak nejlépe sdělit sousední stanici, že má počkat?
\solution{Napsat zprávu, zavolat, případně nechat žádost minutu pípat.
Zvuk žádosti lze dočasně potlačit klikem na ikonu v~horním panelu.}

\item Jak poznáte dopravní a~manipulační kolej?
\solution{Manipulační kolej je ohraničena seřaďovacími návěstidly. Pozor:
přerušovaný symbol koleje na reliéfu není manipulační kolej! Přerušovaný symbol
koleje na reliéfu je kolej bez indikace obsazení.}

\item Jak poznáte směr trati na reliéfu?
\solution{Ze směru šipky úvazky.}

\item Úvazka je červená, co to znamená?
\solution{Nastala porucha blokové podmínky trati.}

\end{enumerate}

\subsubsection*{Jízdní cesty}
\begin{enumerate}[leftmargin=*]

\item Popište vztah mezi pojmy \textit{jízdní cesta}, \textit{vlaková cesta}
a~\textit{posunová cesta}.
\solution{Jízdní cestou rozumíme cestu pro vlak (vlaková cesta) nebo pro posun
(posunová cesta).}

\end{enumerate}

\subsubsection*{Krizové scénáře}
\begin{enumerate}[leftmargin=*]
\item Ohlédnete se do sousední stanice
a~vidíte, že dva vlaky jedou proti sobě a~nezpomalují, přibližně za 5 s~do sebe
narazí. Co uděláte?
\solution{Klik na červenou ikonku na horní liště panelu \textit{Zastavit DCC
na celém kolejišti}. Poté otevřít ovladač obou souprav, zastavit je a pustit
DCC. Tím lokomotivy zůstanou stát a~celé kolejiště se rozjede.}

\item Kdy použít nouzové zastavení celého kolejiště a~kdy vybrané soupravy?
\solution{Nouzové zastavení celého kolejiště používejte v~případě, kdy není
možné vlak zastavit jinak. Typicky například proto, že se vlak nenachází
v~obvodu vámi řízené stanice. Nouzové zastavení celého kolejiště je krajní
volbou!}

\end{enumerate}

\subsubsection*{Odpovědnost}
\begin{enumerate}[leftmargin=*]
\item Popište, za co jako dispečer odpovídáte.
\solution{Za všechny soupravy v~obvodu vámi řízených stanic – za to, že se
nepoškodí.}

\item Co dělat, když na kolejišti něco rozbijete?
\solution{Nahlásit starší obsluze.}

\end{enumerate}

\subsubsection*{Vlakotvorba}
\begin{enumerate}[leftmargin=*]
\item Jaké jsou základní režimy fungování kolejiště z~hlediska vlakotvorby
a oběhů vlaků?
\solution{Neježdění nákladní dopravy × ježdění nákladní dopravy. \\
Ježdění v~taktu × provoz podle grafikonu.}

\item Popište, podle čeho se rozhodujete, jaký vlak kdy a~kam poslat.
\solution{Průběžně sledujete stav v~sousedních stanicích a~snažíte se jim vyjít
vstříc. Pokud mají plno, neposíláte vlaky a~místo toho třeba posunujete. Pokud
se do sousední stanice chystá osobní vlak, pošlete taky osobní, abyste
modelovali přípoje atp. Ideální je si na začátku směny se sousedními stanicemi
dohodnout takt (např. \uv{jeden vlak za jeden vlak}), případně jezdit podle
grafikonu.}

\end{enumerate}

\subsubsection*{Další}
\begin{enumerate}[leftmargin=*]
\item Co je to riziková funkce?
\solution{Jedná se o~rizikové operace, jako například rušení vlaku,
přestavování obsazené výhybky, uvolňování závěru. Při průběhu rizikové funkce
je zobrazeno speciální okno, kde je třeba operaci ještě jednou potvrdit.}

\item Které všechny úkony je nutné vykonat při potvrzování rizikové funkce?
Uveďte na příkladu nouzového stavění výhybky.
\solution{Je nutné přečíst si potvrzovanou rizikovou funkci a výpis
kontrolovaných podmínek. Občas se ve výpisu může objevit čekání na nějakou
akci (uzavření přejezdu / přestavení výhybky), v~takovém případě je nutné
počkat na dokončení těchto úkonů. A~poté odsouhlasit seznam kontrolovaných
podmínek.}

\end{enumerate}


\newpag
\end{enumerate}

\end{document}
